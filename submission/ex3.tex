\documentclass[a4paper, 12pt]{article}
\usepackage[margin=2.5cm]{geometry}
\usepackage{amsmath}
\usepackage{amssymb}

\usepackage{setspace}
\onehalfspacing
\setlength{\parindent}{0pt}

\emergencystretch=\maxdimen
\hyphenpenalty=10000

\begin{document}
    Boqian Yu 03708925\\
	
    \textbf{Derivation:}
    
    Let $P_L \triangleq O_L X$ and $P_R \triangleq O_R X$, we have:
    $$P_R = R P_L + t,$$
    which implies that $P_R$ and $RP_L+t$ are colinear. We have:
    \begin{equation*}
    \begin{aligned}
        0 = (R P_L + t) \times P_R&= (R P_L) \times P_R + t \times P_R \\
         &= (RP_L) \times P_R + t \times (RP_L + t) \\
         &= (RP_L) \times P_R + t \times (RP_L) \\
        \text{multiply $P_R^T$ on both sides} \\
        0&= P_R^T \cdot \left((R P_L) \times P_R \right) + P_R^T \cdot \left( t \times (RP_L) \right) \\
         &= P_R^T \cdot \left( t \times (RP_L) \right) \\
         &= P_R^T \cdot [t]_{\times} R \cdot P_L
    \end{aligned}
    \end{equation*}
    where $[t]_{\times} \cdot (R P_L) = t \times (R P_L)$.
    
    Going back to the definition of essential matrix, we have $E = [t]_{\times}R$
\end{document}